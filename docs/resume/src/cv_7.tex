
\documentclass[11pt,a4paper,sans]{moderncv} % Font sizes: 10, 11, or 12; paper sizes: a4paper, letterpaper, a5paper, legalpaper, executivepaper or landscape; font families: sans or roman
\usepackage{standalone}
\moderncvstyle{classic} % CV theme - options include: 'casual' (default), 'classic', 'oldstyle' and 'banking'
\moderncvcolor{blue} % CV color - options include: 'blue' (default), 'orange', 'green', 'red', 'purple', 'grey' and 'black'
\usepackage{multicol}
\usepackage{lipsum} % Used for inserting dummy 'Lorem ipsum' text into the template
\usepackage[hyperref]{}
\usepackage{fontawesome}
\usepackage{ amssymb }
\usepackage[scale=0.9]{geometry} 
\usepackage{bibentry}
%----------------------------------------------------------------------------------------
%	NAME AND CONTACT INFORMATION SECTION
%----------------------------------------------------------------------------------------

\firstname{Utkarsh} % Your first name
\familyname{Patel} % Your last name

%\address {Jaymala Nagar,Lane No-3,}{Old Sangvi, Pune- 411027}
%\mobile{+91-$\blacksquare\blacksquare\blacksquare\blacksquare\blacksquare$-$\blacksquare\blacksquare\blacksquare\blacksquare\blacksquare$ }
\mobile{+91-95476-21111}
\email {imutkarshpatel@gmail.com}
\begin{document}

\makecvtitle % Print the CV title
%-----------------------------------------------------
%	EDUCATION SECTION
%-----------------------------------------------------
\section{Education}

\cventry{May 2023}{Indian Institute of Technology Kharagpur}{Kharagpur, India}{\newline BTech and MTech in Electronics and Electrical Communication Engineering | CGPA 9.47/10}{\newline Minor in Computer Science and Engineering}{} 


% \cventry{May 2017}{Shah Faiz Public School}{Ghazipur, India}{\newline \small{All India Senior School Certificate Examination | Aggregate 94.8\%}}{}{} 

% \cventry{May 2015}{Shah Faiz Public School}{Ghazipur, India}{\newline \small{All India Secondary School Examination | CGPA 10/10}}{}{}

\section{Work Experience}

\cventry{May 2022 -- Jul 2022}{Software Engineer Intern}{\textsc{The D. E. Shaw Group}}{{\small \textit{Hyderabad, India}}}{}{
\begin{itemize}
\item Worked in the Front Office R\&D Tech division on firm's \textbf{core analytical engine} for discretionary strategies
\item Proposed and implemented an \textbf{end-to-end} feature to support searching, filtering and on-the-fly vectorized computation of complex symbolic expressions over cached time-series data for different instrument attributes
\item Created optimized \textbf{RESTful APIs} for exposing backend functionality to \textbf{Tableau} web data connector, allowing traders to analyze custom time-series in Tableau server
\end{itemize}}

\section{Publications}

\cventry{2023}{}{}{\textbf{Utkarsh Patel}, Animesh Mukherjee, Mainack Mondal. ``Dummy Grandpa, do you know anything?'': Identifying and Characterizing Ad hominem Fallacy Usage in the Wild.}{\newline \small In \textit{Proceedings of the 17\textsuperscript{th} International AAAI Conference on Weblogs and Social Media (ICWSM '23).}}{}







\section{Research Experience}

\cventry{Feb 2021 -- Ongoing}{Ad hominem Fallacies in the Wild \href{https://github.com/utkarsh512/adhominem}{\faGithub}}{}{\small{| \textit{Guides: Prof. Mainack Mondal and Prof. Animesh Mukherjee}}}{}{
\begin{itemize}
\item Implemented explainable models to detect \textbf{ad hominem} fallacies and provide linguistic insight into their triggers
\item Achieved state-of-the-art results on sparsely annotated datasets using \textbf{SS-GAN} schema applied over \textbf{BERT}
\item Performed network studies on the users to understand user dynamics in debate portals and social media sites
\item Validated our in-the-wild predictions by performing crowdsourced surveys, achieving macro-F1 up to \textbf{0.94}
\end{itemize}}

\cventry{Aug 2020 -- Dec 2020}{Detection of Autism Spectrum Disorder \href{https://github.com/utkarsh512/Autism}{\faGithub}}{}{\small{| \textit{Guide: Prof. Debasis Samanta}}}{}{
\begin{itemize}
\item Worked on the \textbf{ABIDE} dataset to extract and process resting-state functional MRI data using \textbf{nilearn} 
\item  Used correlation-based approach to determine functional connectivity between regions of interest
\item Achieved test accuracy of \textbf{0.68} and \textbf{0.65} using \textbf{Support Vector Machines} and \textbf{K-Nearest Neighbors}
\end{itemize}}





\section{Achievements}
\cvitem{2022}{\textbf{Department Rank 1} among the Dual Degree (VIPES) students of the Department of E\&ECE}
\cvitem{2021}{Secured \textbf{Global Rank 70} among 12,000+ contestants in \textbf{Google Kick Start} (Round C)}
\cvitem{2021}{Qualified for Round 2 of \textbf{Facebook Hacker Cup}}
% \cvitem{2019}{Secured \textbf{department change} to E\&ECE by acquiring \textbf{9.69} CGPA at the end of the first year}




\section{Projects}

\cventry{Spring 2022}{Defog: Single Image Defogging by Multiscale Depth Fusion \href{https://github.com/utkarsh512/defog}{\faGithub}}{}{\small{| \textit{Computer Vision}}}{}{
\begin{itemize}
\item Implemented inhomogeneous \textbf{Laplacian–Markov} random field regularized with smoothing and edge-preservation
\item Used \textbf{max-flow min-cut} algorithm for energy minimization with \textbf{alpha-beta swap} for depth map estimation
\end{itemize}}

\cventry{Winter 2021}{Facebook Scraper \href{https://github.com/utkarsh512/fbscraper}{\faGithub}}{}{\small{| \textit{Web Scraping}}}{}{
\begin{itemize}
\item Developed a web crawler to scrape posts, comments and replies from public Facebook pages
\item Used \textbf{selenium} to automate the browsing and \textbf{Beautiful Soup} for parsing the page source
\end{itemize}}

\cventry{Fall 2021}{Jarvis: Chatbot for Customer Support \href{https://github.com/utkarsh512/jarvis}{\faGithub}}{}{\small{| \textit{Natural Language Processing}}}{}{
\begin{itemize}
\item Implemented a \textbf{Seq2Seq} architecture based chatbot with \textbf{Luong} attention mechanism in \textbf{PyTorch}
\item Trained the model on \textbf{Customer Support on Twitter} dataset with \textbf{teacher forcing} and \textbf{gradient clipping}
\end{itemize}}

% \cventry{Fall 2021}{User Authentication using Keystroke Dynamics \href{https://github.com/utkarsh512/User-Authentication-using-Keystroke-Dynamics}{\faGithub}}{}{\small{| \textit{Machine Learning}}}{}{
% \begin{itemize}
% \item Implemented an \textbf{Artificial Neural Network} to authenticate users using keystroke dynamics of their mood data 
% \item Extracted hold time and latency values for different keys and used them as feature vectors for classification
% \end{itemize}}

% \cventry{Spring 2021}{Create-Debate Scraper \href{https://github.com/utkarsh512/CreateDebateScraper}{\faGithub}}{}{\small{| \textit{Web Scraping}}}{}{
% \begin{itemize}
% \item Developed a web crawler to scrape all the debates from CreateDebate.com using \textbf{Beautiful Soup} 
% \item Used \textbf{NetworkX} to construct graphs representing the nested structure of the comments in the threads
% \end{itemize}}

% \cventry{Winter 2020}{Targeted Aspect-based Sentiment Analysis \href{https://github.com/utkarsh512/ABSA-Bert}{\faGithub}}{}{\small{| \textit{Natural Language Processing}}}{}{
% \begin{itemize}
% \item Transformed the task to sentence-pair classification by constructing auxiliary sentences from target-aspect pairs
% \item Fine-tuned \textbf{BERT} on \textbf{SentiHood} dataset, achieved aspect F1-score \texbf{0.90} and sentiment AUC \textbf{0.98}
% \end{itemize}}

\iffalse
\cventry{Fall 2020}{Learning Algorithms \href{https://github.com/utkarsh512/CS60050-Machine-Learning/tree/master/Assignment}{\faGithub}}{}{\small{| \textit{Machine Learning}}}{}{
\begin{itemize}
\item Implemented \textbf{Decision Trees} to predict the increase in Covid-19 cases, used \textbf{RE-pruning} to prevent over-fitting
\item Implemented \textbf{Naive Bayes} classifier to predict patient length-of-stay in ICUs, used \textbf{PCA} for dimension reduction
\item Used \textbf{SVM} and \textbf{MLP} classifier to predict the biodegradability of chemical from its molecular description
\end{itemize}}
\fi


\section{Technical Skills}

\cvitem{Languages}{C/C++ (Proficient), Python (Proficient), JavaScript, Bash, SQL}
\cvitem{Frameworks}{Pandas, NumPy, PyTorch, Scikit-learn, TensorFlow, Tableau WDC}

% \section{Relevant Coursework}

% \cvitem{}{Algorithms, Computer Architecture, Operating Systems, Computer Networks, Database Management Systems, Probability and Stochastic Processes, Machine Learning, Natural Language Processing}


%-----------------------------------------------------
%	COMPUTER SKILLS SECTION
%----------------------------------------------------



%----------------------------------------------------------------------------------------
%	COMMUNICATION SKILLS SECTION
%----------------------------------------------------------------------------------------
\end{document}